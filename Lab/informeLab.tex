\documentclass[11pt, a4paper]{article}

% --- PREÁMBULO UNIVERSAL ---
\usepackage[top=2.5cm, bottom=2.5cm, left=2.5cm, right=2.5cm]{geometry}
\usepackage{fontspec}

% Configuración de idioma (Español) para LuaLaTeX
\usepackage[spanish, bidi=basic, provide=*]{babel}

% Personalización de nombres (Tabla en vez de Cuadro)
\addto\captionsspanish{
    \renewcommand{\tablename}{Tabla}
    \renewcommand{\listtablename}{Índice de tablas}
}

% Proveer idiomas adicionales
\babelprovide[import, onchar=ids fonts]{english}
\babelprovide[import, onchar=ids fonts]{japanese}

% --- CONFIGURACIÓN DE FUENTES (TIPOGRAFÍA) ---
% Usamos 'Latin Modern Roman' (Computer Modern moderna).
% Es la fuente estándar científica/académica de LaTeX.
\babelfont{rm}{Latin Modern Roman}
% Para japonés usamos Noto Serif (estilo clásico compatible)
\babelfont[japanese]{rm}{Noto Serif CJK JP}

% Configuración de listas
\usepackage{enumitem}
\setlist[itemize]{label=-}

% Formato de párrafos
\usepackage{parskip} 
\usepackage{setspace}
\onehalfspacing

% --- PAQUETES GRÁFICOS Y TABLAS ---
\usepackage{booktabs}
\usepackage{graphicx}
\usepackage{float} % Necesario para la opción [H] en figuras

% --- ENLACES ---
\usepackage{hyperref}
\hypersetup{
    colorlinks=true,
    linkcolor=blue,
    filecolor=magenta,      
    urlcolor=blue,
    pdftitle={Informe Técnico LDAP Docker},
    pdfpagemode=FullScreen,
}

% --- CÓDIGO FUENTE ---
\usepackage{listings}
\usepackage{xcolor}

\definecolor{codegreen}{rgb}{0,0.6,0}
\definecolor{codegray}{rgb}{0.5,0.5,0.5}
\definecolor{codepurple}{rgb}{0.58,0,0.82}
\definecolor{backcolour}{rgb}{0.95,0.95,0.92}

\lstdefinestyle{mystyle}{
    backgroundcolor=\color{backcolour},   
    commentstyle=\color{codegreen},
    keywordstyle=\color{magenta},
    numberstyle=\tiny\color{codegray},
    stringstyle=\color{codepurple},
    basicstyle=\ttfamily\footnotesize,
    breakatwhitespace=false,         
    breaklines=true,                 
    captionpos=b,                    
    keepspaces=true,                 
    numbers=left,                    
    numbersep=5pt,                  
    showspaces=false,                
    showstringspaces=false,
    showtabs=false,                  
    tabsize=2,
    frame=single,
    inputencoding=utf8,
    extendedchars=true,
    literate={á}{{\'a}}1 {é}{{\'e}}1 {í}{{\'i}}1 {ó}{{\'o}}1 {ú}{{\'u}}1 {ñ}{{\~n}}1
}
\lstset{style=mystyle}

% --- DATOS DEL DOCUMENTO ---
\title{\textbf{INFORME TÉCNICO: \\ Migración y Orquestación de Infraestructura LDAP mediante Docker}}
\author{XiaoLong Ji}
\date{28 de Enero, 2026}

\begin{document}

\maketitle
\thispagestyle{empty}

\begin{center}
    \large
    \textbf{Asignatura:} Administración y Mantenimiento de Sistemas y Aplicaciones (AMSA)
\end{center}

\vspace{2cm}

% ÍNDICE DE CONTENIDOS
\tableofcontents
\newpage
\setcounter{page}{1}

% -----------------------------------------------------------------------------
\section{Resumen Ejecutivo y Alcance}

El presente informe documenta el diseño, despliegue y validación de una infraestructura de autenticación centralizada basada en el protocolo LDAP (\textit{Lightweight Directory Access Protocol}).

El proyecto representa una evolución tecnológica respecto a prácticas anteriores realizadas en \textbf{AWS EC2}, migrando de un modelo \textbf{IaaS} (\textit{Infrastructure as a Service}) basado en máquinas virtuales y configuración manual, a un modelo de \textbf{Microservicios} orquestados mediante \textbf{Docker}. El objetivo central ha sido encapsular la lógica de negocio de una autenticación corporativa en contenedores inmutables, garantizando la portabilidad y la persistencia de datos.

% -----------------------------------------------------------------------------
\section{Antecedentes y Análisis de Migración (AWS vs Docker)}

\subsection{El Referente: Despliegue en AWS (Imperativo)}
En la práctica anterior de AWS, el despliegue se basaba en un enfoque \textbf{imperativo}. Se provisionaban instancias EC2 con Amazon Linux y se ejecutaban scripts secuenciales (Bash) para:

\begin{enumerate}
    \item Descargar código fuente en C.
    \item Instalar compiladores (\texttt{gcc}, \texttt{make}) y dependencias del sistema.
    \item Compilar e instalar OpenLDAP manualmente.
    \item Configurar \texttt{systemd} para la gestión del servicio.
\end{enumerate}

\subsection{El Desafío: Adaptación del Script Bash Legacy}
Para esta práctica, se proporcionó un script en Bash diseñado originalmente para el entorno AWS/VM. Durante la fase de análisis, se identificó un \textbf{conflicto conceptual crítico}:

\begin{itemize}
    \item \textbf{El Problema:} El script Bash realizaba la instalación desde cero (compilación). Intentar replicar este script línea por línea dentro de un \texttt{Dockerfile} es considerado un \textit{anti-patrón} en Docker. Hubiera resultado en una imagen pesada, tiempos de construcción excesivos y una gestión ineficiente de capas.
    
    \item \textbf{La Solución Estratégica:} Se adoptó una estrategia de \textbf{"Extracción de Datos y Configuración"}. En lugar de usar el script para instalar el software, se analizó el script para extraer únicamente la \textbf{lógica de negocio} (Estructura del DIT, usuarios 'Jordi' y 'Manel', UIDs 4000/4001 y contraseñas hash).
    
    \item \textbf{Cambio de Paradigma:} Se pasó de \textit{construir el software} (AWS) a \textit{configurar una imagen existente} (Docker), aprovechando la imagen \texttt{osixia/openldap} que ya contiene los binarios compilados y optimizados.
\end{itemize}

% -----------------------------------------------------------------------------
\section{Arquitectura y Diseño Técnico}

La solución se diseñó siguiendo el principio de responsabilidad única, separando la infraestructura en tres servicios definidos en \texttt{docker-compose.yml}.

\subsection{Estrategia de Persistencia y Bootstrapping (\texttt{bootstrap.ldif})}
Para garantizar que el contenedor sea efímero pero los datos persistentes, se utilizó un mecanismo de \textit{bootstrapping}.

\begin{itemize}
    \item \textbf{Qué es:} Un archivo LDIF (\textit{LDAP Data Interchange Format}) que define el estado inicial del directorio.
    \item \textbf{Diseño:} Se tradujeron las variables del script Bash original a entradas LDIF estáticas.
    \begin{itemize}
        \item \textbf{Hash SSHA512:} Se reutilizó el hash \texttt{\{SSHA512\}...} del script original para mantener la consistencia de credenciales sin necesidad de instalar herramientas de generación de hash en el contenedor.
        \item \textbf{Mapeo de IDs:} Se forzaron los \texttt{uidNumber} (4000, 4001) y \texttt{gidNumber} (5000, 5001) específicos requeridos por la lógica de la práctica anterior, asegurando que si se montaran volúmenes de usuario, los permisos de Linux coincidieran.
    \end{itemize}
\end{itemize}

\subsection{Construcción de Imagen Personalizada (\texttt{Dockerfile})}
Se optó por crear una imagen personalizada en lugar de usar la imagen base "vanilla" para garantizar la \textbf{Inmutabilidad de la Configuración}.

\begin{itemize}
    \item \textbf{Base Image:} \texttt{osixia/openldap:1.5.0} (Se evitó el tag \texttt{:latest} para asegurar estabilidad y reproducibilidad).
    \item \textbf{Inyección de Configuración:}
\end{itemize}

\begin{lstlisting}[language=bash, caption=Extracto del Dockerfile]
COPY bootstrap.ldif /container/service/slapd/assets/config/bootstrap/ldif/50-bootstrap.ldif
\end{lstlisting}

Esta línea es crítica. Aprovecha los \textit{hooks} de inicio de la imagen base. Al arrancar, el contenedor detecta este archivo y puebla la base de datos automáticamente. Esto elimina la necesidad de ejecutar comandos \texttt{ldapadd} manuales post-despliegue, automatizando el aprovisionamiento.

\begin{itemize}
    \item \textbf{Variables de Entorno:} Se definieron \texttt{LDAP\_DOMAIN} y \texttt{LDAP\_BASE\_DN} dentro del Dockerfile para "quemar" la configuración de dominio (\texttt{amsa.udl.cat}) en la imagen.
\end{itemize}

\subsection{Orquestación de Servicios (\texttt{docker-compose.yml})}
El orquestador define la topología de red y volúmenes.

\begin{itemize}
    \item \textbf{Red (\texttt{ldap-net}):} Se configuró una red tipo \textit{bridge}. Esto permite \textbf{Service Discovery} automático mediante DNS interno. El contenedor LAM puede encontrar al servidor LDAP simplemente resolviendo el hostname \texttt{ldap-server}, sin gestionar direcciones IP.
    
    \item \textbf{Servicio LDAP (Backend):}
    \begin{itemize}
        \item Expone puertos 389/636.
        \item Monta volúmenes persistentes (\texttt{ldap\_data}) para \texttt{/var/lib/ldap}. Esto asegura que si el contenedor se destruye, la base de datos MDB sobrevive.
    \end{itemize}
    
    \item \textbf{Servicio LAM (Frontend):}
    \begin{itemize}
        \item Versión fijada a \texttt{9.0} (requisito legacy).
        \item Configurado mediante variables de entorno (\texttt{LDAP\_SERVER=ldap-server}) para autoconfigurarse al inicio, evitando la configuración manual vía GUI en cada despliegue.
    \end{itemize}
\end{itemize}

% -----------------------------------------------------------------------------
\section{Implementación y Validación}

\subsection{Despliegue}
El despliegue se realizó mediante el comando:

\begin{lstlisting}[language=bash]
docker-compose up -d --build
\end{lstlisting}

El flag \texttt{--build} forzó la lectura del \texttt{Dockerfile} local, integrando el archivo \texttt{bootstrap.ldif} en la nueva imagen.

\subsection{Pruebas de Autenticación}
Se validó el funcionamiento utilizando un contenedor cliente (\texttt{ubuntu}) dentro de la misma red Docker, simulando un entorno de producción donde el cliente y el servidor están aislados de internet pero conectados entre sí.

\textbf{Prueba de Consulta (Search Request):} \\
Se verificó la existencia del usuario 'Jordi' definido en el script Bash original.

\begin{lstlisting}[language=bash]
ldapsearch -x -H ldap://ldap-server \
-b "dc=amsa,dc=udl,dc=cat" \
-D "cn=admin,dc=amsa,dc=udl,dc=cat" \
-w adminpassword "(uid=jordi)"
\end{lstlisting}

\textbf{Resultado:} \\
El servidor retornó exitosamente el objeto DN \texttt{uid=jordi,ou=users...} con \texttt{uidNumber: 4000}. Esto confirma que:
\begin{enumerate}
    \item El contenedor levantó correctamente el servicio \texttt{slapd}.
    \item El proceso de \textit{bootstrapping} leyó e importó el archivo LDIF correctamente.
    \item La autenticación (Bind) del administrador funcionó.
\end{enumerate}

\subsection{Evidencia Gráfica (LAM)}
Se verificó visualmente la disponibilidad del servicio y la correcta creación de usuarios y grupos a través de la interfaz web LAM en \texttt{localhost:8080}.

% FIGURA 1: LOGIN
\begin{figure}[H]
  \centering
  % --- PASO 1: Descomenta la línea de abajo en tu ordenador ---
  % \includegraphics[width=0.85\textwidth]{imagen.png}
  
  % --- PASO 2: Comenta o borra el bloque \framebox de abajo ---
  \framebox{\parbox{0.8\textwidth}{\centering \vspace{1.5cm} \textbf{IMAGEN: imagen.png} \\ \small (Login de LAM) \vspace{1.5cm}}}
  
  \caption{Pantalla de inicio de sesión de LAM (Evidencia de servicio activo)}
  \label{fig:lam-login}
\end{figure}

% FIGURA 2: USUARIOS
\begin{figure}[H]
  \centering
  % --- PASO 1: Descomenta la línea de abajo en tu ordenador ---
  % \includegraphics[width=0.85\textwidth]{image-lam-user-lab.png}
  
  % --- PASO 2: Comenta o borra el bloque \framebox de abajo ---
  \framebox{\parbox{0.8\textwidth}{\centering \vspace{1.5cm} \textbf{IMAGEN: image-lam-user-lab.png} \\ \small (Gestión de Usuarios) \vspace{1.5cm}}}
  
  \caption{Pantalla de gestión de usuarios (Jordi y Manel)}
  \label{fig:lam-user}
\end{figure}

% FIGURA 3: GRUPOS
\begin{figure}[H]
  \centering
  % --- PASO 1: Descomenta la línea de abajo en tu ordenador ---
  % \includegraphics[width=0.85\textwidth]{image-lam-groups-lab.png}
  
  % --- PASO 2: Comenta o borra el bloque \framebox de abajo ---
  \framebox{\parbox{0.8\textwidth}{\centering \vspace{1.5cm} \textbf{IMAGEN: image-lam-groups-lab.png} \\ \small (Gestión de Grupos) \vspace{1.5cm}}}
  
  \caption{Pantalla de gestión de grupos}
  \label{fig:lam-groups}
\end{figure}

% -----------------------------------------------------------------------------
\section{Conclusiones}

La transición de la práctica de AWS a Docker ha demostrado las ventajas del modelo declarativo. A continuación se presenta una comparativa de impacto entre ambos modelos:

\subsection{Tabla Comparativa}

\begin{table}[H]
\centering
\begin{tabular}{@{}lll@{}}
\toprule
\textbf{Criterio} & \textbf{Script Bash (Legacy)} & \textbf{Docker (Moderno)} \\ \midrule
\textbf{Tiempo de Despliegue} & $\sim$20 minutos & $\sim$1 minuto \\
\textbf{Configuración} & Manual / Imperativa & Automática / Declarativa \\
\textbf{Gestión SSL} & Manual & Automática \\
\textbf{Reproducibilidad} & Poco reproducible & Totalmente reproducible \\ \bottomrule
\end{tabular}
\caption{Comparativa AWS vs Docker}
\label{tab:comparativa}
\end{table}

Además de las mejoras cuantitativas reflejadas en la Tabla \ref{tab:comparativa}, se destacan las siguientes ventajas cualitativas:

\begin{enumerate}
    \item \textbf{Reducción de Complejidad:} Se eliminaron más de 50 líneas de comandos de instalación del script Bash original, reemplazándolas por 5 líneas en un \texttt{Dockerfile}.
    \item \textbf{Idempotencia:} A diferencia del script Bash, que podía fallar si se ejecutaba dos veces (intentando crear usuarios ya existentes), el entorno Docker se reinicia siempre en un estado conocido y limpio.
    \item \textbf{Aislamiento:} Las dependencias de OpenLDAP no contaminan el sistema operativo anfitrión, residiendo únicamente dentro de la imagen del contenedor.
\end{enumerate}

Este laboratorio confirma la adquisición de competencias en la orquestación de servicios seguros y la capacidad de traducir requisitos de infraestructura tradicional a arquitecturas modernas de contenedores.

\vspace{1cm}

\textbf{Firma:} XiaoLong Ji

\textbf{Repositorio del Proyecto:} \url{https://github.com/xji650/practica-ldap.git}

\end{document}